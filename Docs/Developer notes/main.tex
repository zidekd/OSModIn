% SPDX-FileCopyrightText: 2023 2023 Daniel Židek <danielzidek@post.cz>
%
% SPDX-License-Identifier: GPL-3.0-or-later

\documentclass[12pt]{article}

\usepackage[utf8]{inputenc}
\usepackage[T1]{fontenc}

\usepackage{geometry}
\geometry{
    a4paper,
    total={150mm, 247mm},
    left={30mm},
    top={25mm}
}
\usepackage[english,czech]{babel}
\usepackage{mathpazo}
\usepackage{graphicx}
\usepackage{hyperref}
\usepackage{wrapfig}
\usepackage[
    backend=bibtex,
    style=iso-numeric,
    currentlang=true,
    sorting=ynt
]{biblatex} %Imports biblatex package
\addbibresource{citations.bib} %Import the bibliography file
\usepackage[automake, acronym, toc]{glossaries}
\graphicspath{ {./attachments} }

\makeatletter
\renewcommand{\fnum@figure}{Obr. \thefigure}
\makeatother

\makeglossaries

\renewcommand{\contentsname}{Obsah}
\renewcommand{\listfigurename}{Seznam obrázků}
\renewcommand{\listtablename}{Seznam tabulek}
\renewcommand{\refname}{Zdroje}

\title{OpenOscilloscopeModule}
\author{Daniel Židek}
\date{Září 2023}

\begin{document}

%\newacronym{label}{acronym}{whole phrase}
%           [longplural={whole phrases},shortplural=labels]

%\acrshort{lcd}      - LCD
%\acrlong{lcd}       - Liquid Crystal Display
%\acrfull{lcd}       - Liquid Crystal Display (LCD)

%\acrshortpl{lcd}    - LCDs
%\acrlongpl{lcd}     - Liquid Crystal Displays
%\acrfullpl{lcd}     - Liquid Crystal Displays (LCDs)


\newacronym{lcd}{LCD}{Liquid Crystal Display}
\newacronym[longplural={Časové Základny},shortplural=ČZ]{casz}{ČZ}{Časová Základna}
\newglossaryentry{trigger}
{
    name=trigger,
    description={Spouštěč, systém odpovědný za spuštění \acrlongpl{casz}}
}

\newglossaryentry{i2c}
{
    name=I²C,
    description={Inter-Integrated Circuit je sériová sběrnice používaná nejčastěji u
    nízkorychlostních periferií}
}

% SPDX-FileCopyrightText: 2023 2023 Daniel Židek <danielzidek@post.cz>
%
% SPDX-License-Identifier: GPL-3.0-or-later

\begin{titlepage} % Suppresses displaying the page number on the title page and the subsequent page counts as page 1
	\newcommand{\HRule}{\rule{\linewidth}{0.5mm}} % Defines a new command for horizontal lines, change thickness here
	
	\center % Centre everything on the page
	
	%------------------------------------------------
	%	Headings
	%------------------------------------------------
	
	\textsc{\LARGE Daniel Židek}\\[1.5cm] % Main heading such as the name of your university/college
	
	\textsc{\Large OSModIn - Modulární laboratorní přístroje}\\[0.5cm] % Major heading such as course name
	
	\textsc{\large Technická dokumentace}\\[0.5cm] % Minor heading such as course title
	
	%------------------------------------------------
	%	Title
	%------------------------------------------------
	
	\HRule\\[0.4cm]
	
	{\huge\bfseries Zařízení modulární instrumentace}\\[0.4cm] % Title of your document
	
	\HRule\\[1.5cm]
	
	%------------------------------------------------
	%	Author(s)
	%------------------------------------------------
	
	\begin{minipage}{0.4\textwidth}
		\begin{flushleft}
			\large
			\textit{Autor}\\
			Daniel \textsc{Židek} % Your name
		\end{flushleft}
	\end{minipage}
	~
	\begin{minipage}{0.4\textwidth}
		\begin{flushright}
			\large
			\textit{Vedoucí práce}\\
			Daniel \textsc{Židek} % Supervisor's name
		\end{flushright}
	\end{minipage}
	
	% If you don't want a supervisor, uncomment the two lines below and comment the code above
	%{\large\textit{Author}}\\
	%John \textsc{Smith} % Your name
	
	%------------------------------------------------
	%	Date
	%------------------------------------------------
	
	\vfill\vfill\vfill % Position the date 3/4 down the remaining page
	
	{\large Září 2023} % Date, change the \today to a set date if you want to be precise
	
	%------------------------------------------------
	%	Logo
	%------------------------------------------------
	
	%\vfill\vfill
	%\includegraphics[width=0.2\textwidth]{placeholder.jpg}\\[1cm] % Include a department/university logo - this will require the graphicx package
	 
	%----------------------------------------------------------------------------------------
	
	\vfill % Push the date up 1/4 of the remaining page
	
\end{titlepage}

\setcounter{page}{2}

\tableofcontents

\newpage

\addcontentsline{toc}{section}{Seznam obrázků}
\renewcommand{\listfigurename}{Seznam obrázků}
\listoffigures

\newpage

\addcontentsline{toc}{section}{Seznam tabulek}
\renewcommand{\listtablename}{Seznam tabulek}
\listoftables

\newpage

\section{Časový plán}

\subsection{Q3 2023}

Ve třetím kvartálu by měl být hotový návrh analogového front-end osciloskopu.
To zahrnuje hotový atenuátor, obvod zesilovače a analogově digitálního převodníku.

\subsection{Q4 2023}

V tomto kvartálu by měl být hotov obvod zpracování dat a vymyšlen protokol pro
komunikaci modulů. Zároveň by už mělo být dokončeno šasí a další podpůrné části.

\subsection{Q1 2024}

V tomto kvartálu by měl být dokončen software a řídící modul.

\subsection{Q2 2024}

\subsection{Q3 2024}

\subsection{Q4 2024}

\newpage

\section{Abstrakt}

Tato práce se zabývá návrhem a realizací osciloskopu pod otevřenou licencí CERN-OHL-S v2.
Výsledkem by měl být osciloskop v Eurocard formátu, s šířkou pásma alespoň 50 MHz, 
2 kanály a cenou do 15 tisíc CZK. Tato práce dokumentuje proces vývoje,
rozhodnutí a další poznámky týkající se OSModIn Osciloskopu. Kromě toho zde
dokumentuji i proces seznamování se se zpracováním rychlých analogových
signálů a práce s FPGA - celý projekt vzniká jako záminka naučit se něco nového.
Celý projekt je dostupný na GitHubu:
\href{https://github.com/zidekd/OSModIn}{zidekd/OSModIn}.

\newpage

\section{Úvod}

Celý projekt začínám pro to, abych si zkusil něco nového.
Selhal a zkusil to znovu. Ve výsledku bych rád disponoval systémem
podobným PXI od National Instruments s několika moduly.
Jmenovitě je zatím plánovaný osciloskop a laboratorní zdroj.
Všechny přístroje budou disponovat univerzálním protokolem pro
komunikaci (kód tak bude moct být znovupoužitelný), komunikačními
linkami a především obslužným software pro MS Windows (XP a 10),
GNU+Linux a MacOS. Řídící jednotka bude moct sloužit také jako server,
který by dovoloval obsluhu přístrojů nejen lokálně, přes USB, WiFi či LAN
ale i sdílet přístroje reverzním tunelem. Zde bude otázka zabezpečení,
nicméně takový software by měl dovolovat přístup více uživatelů s
různými oprávněními (číst data z přístrojů, měnit jejich nastavení, ...).
To by mohlo být příhodné ve výuce, či na pracovištích s možností vzdálené práce.

Formát Eurocard jsem vybral pro to, protože mi přijde nejvhodnější.
Je tedy ještě potřeba domyslet použití konektorů a komunikačních protokolů,
ale o tom potom. :) Eurocard standard dovoluje použití modulů, jež jsou
ukládány do šasí které lze následně připevnit do 19 palcového racku. Velikost
racku jsou obvykle násobky tří, tedy 3U, 6U, atd.
\emph{(\uv{U} je standard výšky vybavení do racku. 1U je 1,75 palce, nebo 44,45 mm.)}

Co se obslužného software týče tak by moduly měly podporovat ovládání
skrz Python API, vizuální grafické rozhraní připomínající ovládáním běžných
laboratorních přístrojů a také by měly moct být ovládány skrze vizuální editor 
uzlú (podobný Node-RED, LabVIEW...). Ovládání skrze Python API, GUI i Node-based editor
bude možné jak lokálně přes USB, WiFi a LAN tak i přes reverzní tunel, kdy potom
lze ovládat přístroje i na druhém konci světa. Ačkoliv jde o sdílení "pouze" připojených
přístrojů, všechny aplikace schopné s takovým spojením interagovat budou podporovat
správu oprávnění a logů. Řídící jednotka bude zaznamenávat všechny akce, dle úrovní,
takže v závislosti na nastavení se budou ukládat různě závažné záznamy. Co do správy
oprávnění tak k dispozici bude několik úrovní přístupů a ty půjde omezit jak na
jednotlivé přístroje, tak na celé sestavy.

\newpage

\section{Teoretický úvod}

V následující kapitole rozeberu funkci analogových a digitálních osciloskopů, pokusím
se provést rozbor komerčních osciloskopů (alespoň dle reverzních schémat; u 
otevřených designů je tohle však mnohem jednodušší :D) a také porovnám
návrhy jednotlivých bloků, na základě čehož budu stavit vlastní návrh.

\subsection{Analogové osciloskopy} \label{sec:analog-osci}

Základní princip funkce spočívá ve vychylování elektronového paprsku měřeným
signálem a signálem \acrfullpl{casz}. Vychýlený elektronový paprsek následně
dopadá na speciální stínítko, které se v místě dopadu rozsvěcí. Analogové
osciloskopy se již nepoužívají, ačkoliv pozor, ne každý osciloskop s CRT obrazovkou
je analogový osciloskop. Pokud dnes osciloskop s CRT obrazovkou potkáte,
s největší pravděpodobnopstí jde o analogový paměťový osciloskop, nebo
o \acrfull{dso}. Rozdíly mezi nimi rozeberu později.

\begin{figure}[h]
    \centering
    \includegraphics[width=\textwidth]{analog-oscilloscope-diagram}
    \caption{Blokový diagram analogového osciloskopu (převzato z \cite{HowDoesOscilloscopea})}
    \label{fig:blok-analog-osci}
\end{figure}

\begin{wrapfigure}{r}{0.5\textwidth}
    \centering
    \includegraphics[width=0.5\textwidth]{osci-pasive-probe}
    \caption{Pasivní sonda (převzato z \cite{DeutschStandardTastkopfFur2020})}
    \label{fig:osci-pasive-probe}
\end{wrapfigure}

Na obrázku \ref{fig:blok-analog-osci} jako první vidíme blok "probe", tedy osciloskopickou
sondu. Ta slouží k samotnému měření. Obvykle říkáme pouze "sonda", z kontextu je
totiž jasné, o co jde. Nejčastěji se setkáme s pasivními sondami,
(na obr. \ref{fig:osci-pasive-probe}) to jsou sondy ideální na takové běžné měření
do 500 MHz a pravděpodobně to jsou i sondy, co přišly s vašim osciloskopem.
Dále máme atenuátory, což je část obvodu odpovědná za snížení amplitudy vstupního signálu.
Je-li vstupní signál příliš velký, můžeme využít právě atenuátoru pro snížení
amplitudy bez změny frekvence, či jiného zásahu do samotného tvaru signálu.
Atenuátor se obvykle vyskytuje jen po pár krocích, obvykle 1:4, 1:10 a 1:20.
V tomto blokovém schématu je spolu s atenuátorem spojen ještě zesilovač vertikální osy.
Ten se zase stará o spojitou regulaci signálu, abychom například využili celou výšku
zobrazovadla a to i signálem s nízkou amplitudou. Ze zesilovače signál
jde dále do \gls{trigger}u. Jeden z nejčastějčích obvodů pro spouštěcí systém je Schmittův
klopný obvod. Ten slouží ke spuštění tzv. sweep generátoru, tedy generátoru \acrfullpl{casz}.
Stejně jako u vstupního signálu i signál \acrshort{casz} pokračuje do zesilovače a z něj na
vychylovací destičky pro horizontální osu.

Analogové osciloskopy mají zásadní nevýhodu
a to tu, že měřený průběh nelze zastavit, zpomalit nebo jinak upravovat. Ačkoliv ho lze
zaznamenat například fotoaparátem či kamerou (a ano, tohle byl skutečně způsob jakým se
dlouho průběhy zaznamenávaly) tak jeho analýza je poměrně obtížná a často nepřesná.
Navíc zobrazit pouze periodické průběhy, takže se s nimi nepodíváme třeba ani na
\gls{i2c} komunikační sběrnice, jako jeden z mnoha příkladů s čimž se můžeme v elektrotechnické
laboratoři setkat. Proto se začaly využívat \acrlongpl{dso}.

\subsection{Digitálně paměťový osciloskop}

Právě \acrlongpl{dso} řeší tyto nedostatky analogových osciloskopů. Jejich základní princip
funkce zůstává totožný tomu u analogových osciloskopů, hlavní rozdíl všask spočívá ve zobrazování
a vzorkování. Zatím co analogový osciloskop využívá měřeného signálu pro zobrazení průběhu
vychylováním elektronového paprsku, digitální osciloskopy využívají \acrfull{adc} a
číslicové techniky pro zpracování a zobrazení měřeného průběhu. Takto zachycený signál
lze snáze uložit, analyzovat a jeho hlavní limitací je paměť osciloskopu a vzorkovací frekvence
\acrshort{adc}. Sondu, atenuátor, zesilovač a trigger v téhle části vynechám a odkážu se na
kapitolu analogových osciloskopů - \ref{sec:analog-osci}.

\begin{figure}[h]
    \centering
    \includegraphics[width=\textwidth]{digital-oscilloscope-diagram}
    \caption{Blokový diagram digitálního osciloskopu (převzato z \cite{SampleProcessingDigital})}
    \label{fig:blok-digital-osci}
\end{figure}

Poté co je signál utlumen
a zase zesílen na požadovanou úroveň, pokračuje do \acrshort{adc}, kde se navzorkuje.
Hlavním parametrem \acrshort{adc} je vzorkovací frekvence, kvantizační úrovně a kvantizační
šum. Vzorkovací frekvence \textbf{musí} být alespoň dvakrát větší jak frekvence měřeného
průběhu. Jde o Nyquistovu frekvenci, pokud ji dosáhneme, dochází k úplnému a nevratnému
zkreslení signálu \cite{AliasingADCsNot2015}. $$f_{max}=\frac{f_s}{2}$$ V praxi však vzorkovací
frekvence \acrshort{adc} bývá pěti až deseti násobek šířky pásma. Kvantizační úroveň je údaj
o tom jaký nejmenší napěťový díl jsme schopni zachytit. Například 8-bit \acrshort{adc} je
schopné popsat vstupní napětí až 256 hodnotami. Například \acrshort{adc} s full scale rozsahem
0 až 24V a 8 bity kvantizační úrovně má rozlišení 93,75mV.
$$U_{res}=\frac{U_{FS}}{2^n}=\frac{24}{2^8}=0, 09375 [V]$$ Kvantizační šum je šum zanášený
samotným \acrshort{adc}. Po převedení na digitální signál je tato informace zpracována
procesorovou jednotkou (MCU, CPU, FPGA...), uložena do paměti RAM a poté už je jen vykonstruován
obraz a jeho zobrazení na displeji.

\subsection{Osciloskop s digitálním fosforem (DPO)}

TODO: Dodělat

\subsection{Analogový front-end osciloskopu}

Zde rozeberu návrh obvodu atenuátoru, zesilovače, \acrshort{adc}, triggeru a dalších obvodů
potřebných pro zpracování rychlých analogových signálů. Pokusím se najít ukázky těchto obvodů
v již existujících osciloskopech, provést jejich simulaci, porovnat je a na základě toho
provést návrh vlastních. Cílem by měly být informované rozhodnutí při návrhu vlastního osciloskopu.

\subsubsection{BNC a vstupní impedance, AC/DC/GND coupling}

Použití standardního BNC konektoru je jasný. Za ním však dochází ještě k výběru mezi 50 $\Omega$ a 1 M$\Omega$
vstupní impedancí. \emph{(Pravděpodobně bude v tomto návrhu vynecháno.)} Zde je důležité použít
kvalitní relé, či přepínač. Co se couplingu týče,
DC volba je přímé propojení vstupního signálu na front-end. Pro AC coupling připojíme paralelně cca
100nF kondenzátor a pro GND coupling spojíme analogové obvody se zemí.

\begin{figure}[h]
    \centering
    \includegraphics[width=0.5\textwidth, page=2]{kicad-export}
    \caption{Příklad realizace vstupu osciloskopu}
    \label{fig:input-example}
\end{figure}

Je důležité upozornit že AC coupling nelze použít všude! AC coupling je vhodný pro velké DC offsety,
například když se chceme podívat na zákmity (běžně mohou mít do 100mV) na +12V větvi zdroje.
Se signály jako je ten obdélníkový nám může použitý kondenzátor zkreslit signál a my si toho
nemusíme ani všimnout. Pro menší DC offsety je tak vhodné použít jiných metod odečítání DC
složky průběhu.

\subsubsection{Atenuátor}

Zde amplitudu vhodně podělíme, to proto abychom mohli měřit signály i mimo rozsah \acrshort{adc}.
Ve své podstatě jde pouze o napěťový dělič. V ideálním prostředí bez parazitních parametrů by
stačilo použít pouze rezistory jakožto napěťový dělič, nicméně protože naše cestičky a prvky
mají i parazitní kapacitu, musíme pro ni kompenzovat. Kompenzovat pro ni musíme, protože bez ní
by se lišil poměr děliče kapacitního potenciálu a děliče potenciálu tvořeného rezistory.
\cite{AnswerCompensatedAttenuator2016}

Rezisotry R301, R302 a R303 tvoří standardí napěťový dělič se sériovým odporem 1M$\Omega$. Kondenzátor C304
znázorňuje parazitní kapacitu v obvodu a kondenzátory C301, C302 a C303 slouží právě pro kompenzaci děliče.
Při realizaci je vhodné tyto kondenzátory nahradit (či doplnit) variabilními kapacitními trimry pro
přesné doladění po sestavení. Kompenzace se provadí přivedením obvykle 10kHZ obdélníkového signálu
na vstup a následné nastavení trimerů pro co nejlepší hranu obdélníku. Kompenzovat se musí každý ze
stavů zvlášť, tedy volba 1:1, 1:10 i 1:100.

\begin{figure}[h]
    \centering
    \includegraphics[width=0.75\textwidth, page=3]{kicad-export}
    \caption{Příklad realizace atenuátoru osciloskopu}
    \label{fig:attenuator-example}
\end{figure}

TODO: Doplnit výstupy simulace a další metody útlumu

\subsubsection{Zesilovače}

\newpage

\subsubsection{Analogově digitální převodník}

Analogově digitální převodník slouží k vzorkování analogových signálů a
následném převedení na digitální slova \emph{(slovem myšlen byte)}.
Na obrázku \ref{fig:adc-sampling} lze vidět princip vzorkování. Zelená
čára představuje měřený analogový signál, body na něm jsou naše vzorky a
ty modré spojnice ředstavují čas mezi vzorkováním. Kvantizační úroveň je
parametr který udává výšku "schodů" na grafu. Udáváme ji v bitech a standardní
rozlišení je 8 bitů, tedy jde o 256 kvantizačních úrovní ($2^8$).
Teď už daleko častěji potkáváme i osciloskopy s 12bitovými a 14 bitovými
převodníky, často označované jako \uv{High Definition} nebo \uv{High Resolution}.

\begin{figure}[h]
    \centering
    \includegraphics[width=\textwidth]{adc-sampling}
    \caption{Nákres funkce ADC při vzorkování (převzato z \cite{OscilloscopeBasicsPrimer})}
    \label{fig:adc-sampling}
\end{figure}

Vzorkovací teorém (nyquist-shannonův teorém) říká, že vzorkovací frekvence
musí být dvakrát tak vyšší, než je frekvence nejvyšší harmonické ve fourierovském
spektru měřeného signálu. V praxi volíme ale frekvenci o něco vyšší, pro měření to
bývá i 5 až 10 krát vyšší frekvence vzorkování. \cite{MereniKolemNas} Na obrázku \ref{fig:adc-aliasing}
vidíme 1 GHz sinusový signál vzorkovaný 750 Msps. \uv{Alias} signálu, neboli
jeho chybná rekonstrukce se na grafu prezentuje jako sinus o frekvenci 250 MHz,
což je 4x menší než-li ta skutečná. Pro tyhle případy používáme před převodníkem
filtr typu horní propust, lazený na mezní frekvenci našeho přístroje. Tím zamezíme
takto výrazně chybným měřením a zabráníme tím špatné interpretaci signálu.

\begin{figure}[h]
    \centering
    \includegraphics[width=0.8\textwidth]{adc-aliasing}
    \caption{Nákres ADC vzorkování při aliasingu (převzato z \cite{OscilloscopeBasicsPrimer})}
    \label{fig:adc-aliasing}
\end{figure}

\newpage

\section{Plánované funkce}

Přehle plánované podpory funkcí.

\subsection{Acquisition modes}

\citetitle{OscilloscopeFundamentals} \Citeauthor{OscilloscopeFundamentals}

\begin{description}
    \item [Sample mode] Bod vzorku odpovídá jednomu bodu průběhu
    \item [High Res] Průměr vzorků je zobrazen pro každý interval
    \item [Peak detect] Pro každý interval je zobrazen maximální a minimální vzorek
    \item [RMS] Zobrazí RMS hodnotu vzorků v průběhu
\end{description}

\subsection{Trigger}

\begin{description}
    \item [Rising, falling, rising and falling edge]
    \item [Transit time (Slew rate)]
    \item [Pulse width]
    \item [State (logic)]
    \item [Glitch]
    \item [Time-out]
    \item [Window]
    \item [Setup and Hold]
    \item [Runt pulse]
    \item [Mask violation]
    \item [Patern trigger]
    \item [Protocol trigger]
    \item [A \& B trigger]
\end{description}

% \subsubsection{Schmittův klopný obvod} \label{sec:schmitt-ko}

% TODO: Přesunout do samostatné sekce triggerů, spolu s dalšími obvody

% Schmittův klopný obvod je speciální komparátor, který má hysterezi. To znamená, že jeho
% výstup je závislý nejen na hodnotě vstupu, ale i na jeho původním stavu. Podobně jako
% obyčejný komparátor s operačním zesilovačem, i Schmittův klopný obvod dosahuje na výstupu
% kladného nebo záporného saturačního napětí.

% Pokud je na výstupu například kladné napětí, nedojde k překlopení Schmittova obvodu
% při pouhém splnění podmínky U+ < U- jako u komparátoru, ale teprve až rozdíl obou napětí
% dosáhne prahové hodnoty H- . Podobně, pokud je nyní na výstupu záporné saturační napětí,
% může dojít ke zpětnému překlopení Schmittova obvodu teprve až v momentě, kdy je U+ > U- o
% více než H+. (Celá část \ref{sec:schmitt-ko} je převzata z \cite{ELUCNapetovyKomparator})

% \begin{figure}[h]
%     \centering
%     \includegraphics[width=0.5\textwidth]{schmitt-ko}
%     \caption{Schmittův \acrshort{ko} pomocí \acrshort{nk}}
%     \label{fig:schmitt-ko}
% \end{figure}

\newpage

\printbibliography[
heading=bibintoc,
title={Seznam použité literatury}
]

\newpage

\printglossary[type=\acronymtype, title=Seznam zkratek, toctitle=Seznam zkratek]

\printglossary[title=Slovník pojmů, toctitle=Slovník pojmů]

\end{document}
